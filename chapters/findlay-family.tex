\chapter{Findlay Family Recipes}
\label{chap:findlay-recipes}

\epigraph{
The other night, I ate at a real nice family restaurant.
Every table had an argument going.
}{George Carlin}

\noindent
In this chapter, I will cover some traditional Findlay family recipes --- some
of which date back to well before my father was born! Others are not so traditional,
but have held a special place in my heart for one reason or another. \Cref{sec:findlay-breakfast}
covers breakfast recipes, \cref{sec:findlay-lunch} covers lunch, \cref{sec:findlay-dinner}
covers dinner, and finally \cref{sec:findlay-dessert} covers dessert.

I hope that you will enjoy these recipes. Perhaps some may one day have
a special place in your kitchen as they have had in mine.

\section{Breakfast}
\label{sec:findlay-breakfast}

% =========================================================================== %
% The "William Special"                                                       %
% =========================================================================== %
\begin{recipe}{The \enquote{William Special}}{1 person}{10 min.}
\freeform The \enquote{William Special} is a simple recipe for a breakfast wrap
that I used to make every morning, religiously, for a period of several months
near the beginning of 2020. It gets its name from the \enquote{Grandpa Special},
covered later in this chapter --- my mother jokingly referred to it as such one
fateful morning and the name stuck, a fact I have come to reluctantly accept.

\newstep Heat a medium-sized non-stick skillet over medium-high heat (level
\nicefrac{6}{10} on my stove).  I like to start the kettle boiling for tea while
I wait for the skillet to come up to temperature. You can test whether the pan
is hot enough by wetting one finger with lukewarm water and flicking a drop onto
the pan. If the drop sizzles, the pan is ready.

\Ing{One large flour tortilla}
\Ing{Two slices cheddar cheese}
\Ing{One tsp.~salsa}
Prepare the tortilla by placing two slices of cheese across the center,
length-wise, and spooning the salsa directly next to the row of cheese
slices.

\Ing{Two large eggs}
\Ing{A knob of salted butter}
Place a knob of butter into the hot pan. When the butter has melted, crack
two eggs into the pan and stir vigorously until curds begin to form. When the
eggs are done, quickly dump the contents of the skillet onto your tortilla
and arrange them length-wise to match the cheese.

\newstep Assemble your wrap by folding the ends inwards and rolling
it tightly towards your body. This is a trick I learned by watching the
burrito chefs at Chipotle\texttrademark. Return your tortilla to the pan
and cook until nicely browned, flipping once.
\end{recipe}

\section{Lunch}
\label{sec:findlay-lunch}

% =========================================================================== %
% The "Grandpa Special"                                                       %
% =========================================================================== %
\begin{recipe}{The \enquote{Grandpa Special}}{1 person}{10 min.}
\freeform The \enquote{Grandpa Special} is a bagel sandwich that my grandfather
eats every day for lunch. It's so good that he literally promises to make me one
to convince me to come over and help with computer problems. Yep, it's that
good.  If you don't believe me, try it yourself!  Just remember, this sandwich
\textbf{must} be served with a Strub's Baby Dill Pickle and a can of Coke Zero
---  more on this later.

\Ing{One Montreal-style bagel}
Toast your bagel on X setting (\nicefrac{X}{X} on my grandpa's toaster).
While the bagel is toasting, prepare your other ingredients.

\Ing{Three slices coldcut turkey or roast beef}
\Ing{Two slices provolone cheese}
\Ing{Mayonnaise (or butter)}
Foo.

\newstep Microwave the sandwich for X seconds just before serving. This will slightly
melt the cheese.

\Ing{One Strub's Baby Dill Pickle}
\Ing{One can Coke Zero}
It is considered sacrilege to serve a \enquote{Grandpa Special} without a Strub's
Baby Dill Pickle on the side and a can of Coke Zero. If you can't find Strub's
(it's in the Kosher section of my supermarket), feel free to substitute with
a more or less equivalent brand of pickle.
\end{recipe}

\section{Dinner}
\label{sec:findlay-dinner}

\clearpage

\section{Dessert}
\label{sec:findlay-dessert}

\clearpage
