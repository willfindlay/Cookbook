\chapter{Findlay and Family Recipes}
\label{chap:findlay-recipes}

\epigraph{
The other night, I ate at a real nice family restaurant.
Every table had an argument going.
}{George Carlin}

\noindent
In this chapter, I will cover some traditional Findlay family recipes --- some
of which date back to well before my father was born! Others are not so
traditional, but have held a special place in my heart for one reason or
another. \Cref{sec:findlay-breakfast} covers breakfast recipes,
\cref{sec:findlay-lunch} covers lunch, \cref{sec:findlay-dinner} covers dinner,
and finally \cref{sec:findlay-dessert} covers dessert.

I hope that you will enjoy these recipes. Perhaps some may one day have
a special place in your kitchen as they have had in mine.

\section{Breakfast}
\label{sec:findlay-breakfast}

% =========================================================================== %
% Grandma Findlay's Pancakes                                                  %
% =========================================================================== %
\begin{recipe}{Grandma Findlay's Pancakes}{12 pancakes}{30 min.}
\freeform The recipe for Grandma Findlay's Pancakes has been passed down through
four generations of Findlay's starting with my great-grandmother (hereafter
referred to as "Grandma Findlay") who made them for my grandfather when he was
a boy. After her, my grandmother made them for my father, when he was growing
up, and for my cousin Calvin and me in turn. As far as the recipe's history is
concerned, legend has it that Grandma Findlay copied the recipe off of the back
of a box --- sort of an underwhelming origin story. However, what these pancakes
may lack in originality, they more than make up for in taste and texture, two
properties which have enabled them to stand the familial test of time.

\Ing{1 \fr{1}{2} cups flour}
\Ing{1-2 tbsp.~sugar}
\Ing{3 tsp.~baking powder}
\Ing{\fr{1}{2} tsp.~salt}
Pour dry ingredients into a medium size bowl, whisk to combine.

\Ing{1 large egg, beaten}
\Ing{1 \fr{3}{4} cups milk}
\Ing{2 tbsp.~vegetable oil}
Combine wet ingredients in a separate bowl.

\newstep Add the wet mixture to the bowl of dry ingredients, mixing until just
combined.  If the batter thickens at any point during the process, feel free to
add more milk as needed.

\newstep Heat a frying pan (or griddle) to 380\0F or just until water droplets
dance and sizzle when flicked into the pan.  Grease the pan lightly with cooking
oil and pour \fr{1}{4} cup of mixture onto the pan for each pancake. Cook until
the surface of the pancake is covered with bubbles and flip. Continue cooking
until golden brown.
\end{recipe}

% =========================================================================== %
% The "William Special"                                                       %
% =========================================================================== %
\begin{recipe}{The \enquote{William Special}}{1 serving}{10 min.}
\freeform The \enquote{William Special} is a simple recipe for a breakfast wrap
that I used to make every morning, religiously, for a period of several months
near the beginning of 2020. It gets its name from the \enquote{Grandpa Special},
covered later in this chapter --- my mother jokingly referred to it as such one
fateful morning and the name stuck, a fact I have come to reluctantly accept.

\newstep Heat a medium-sized non-stick skillet over medium-high heat (level
\nicefrac{6}{10} on my stove).  I like to start the kettle boiling for tea while
I wait for the skillet to come up to temperature. You can test whether the pan
is hot enough by wetting one finger with lukewarm water and flicking a drop onto
the pan. If the drop sizzles, the pan is ready.

\Ing{One large flour tortilla}
\Ing{Two slices cheddar cheese}
\Ing{One tsp.~salsa}
Prepare the tortilla by placing two slices of cheese across the center,
length-wise, and spooning the salsa directly next to the row of cheese
slices.

\Ing{Two large eggs}
\Ing{A knob of salted butter}
Place a knob of butter into the hot pan. When the butter has melted, crack
two eggs into the pan and stir vigorously until curds begin to form. When the
eggs are done, quickly dump the contents of the skillet onto your tortilla
and arrange them length-wise to match the cheese.

\newstep Assemble your wrap by folding the ends inwards and rolling
it tightly towards your body. This is a trick I learned by watching the
burrito chefs at Chipotle\texttrademark. Return your tortilla to the pan
and cook until nicely browned, flipping once.
\end{recipe}

\section{Lunch}
\label{sec:findlay-lunch}

% =========================================================================== %
% The "Grandpa Special"                                                       %
% =========================================================================== %
\begin{recipe}{The \enquote{Grandpa Special}}{1 serving}{10 min.}
\freeform The \enquote{Grandpa Special} is a bagel sandwich that my grandfather
eats every day for lunch. It's so good that he literally promises to make me one
to convince me to come over and help with computer problems. Yep, it's that
good.  If you don't believe me, try it yourself!  Just remember, this sandwich
\textbf{must} be served with a Strub's Baby Dill Pickle and a can of Coke Zero
---  more on this later.

\Ing{One Montreal-style bagel}
Toast the bagel lightly. It should be brown on the edges but mostly white.
While the bagel is toasting, prepare your other ingredients.

\Ing{Three slices Montreal-style smoked turkey breast coldcuts}
\Ing{Two slices provolone cheese}
\Ing{Mayonnaise (or butter)}
Lightly apply mayonnaise (or butter) to both sides of the bagel. Place the
cheese on top of the coldcuts and assemble the sandwich.

\newstep Microwave the sandwich for 45 to 50 seconds just before serving.
This will slightly melt the cheese and allow the bagel to soften a bit.
My grandpa likes to squish the bagel down with both hands at this point,
but that step is optional.

\Ing{One Strub's Baby Dill Pickle}
\Ing{One can Coke Zero}
It is considered sacrilege to serve a \enquote{Grandpa Special} without
a Strub's Baby Dill Pickle on the side and a can of Coke Zero. If you can't find
Strub's (it's in the Kosher section of my supermarket), you may begrudgingly
substitute with a more or less equivalent brand of pickle.
\end{recipe}

\section{Dinner}
\label{sec:findlay-dinner}

% =========================================================================== %
% Cheesy Cauliflower                                                          %
% =========================================================================== %
\begin{recipe}{Cheesy Cauliflower Golden Dome}{8 servings}{15 min.}
\freeform This is a killer side dish recipe that is as easy as it is delicious.
Seriously, a monkey could make this. But the monkey probably wouldn't do as good
a job as my grandmother, who would often make this recipe as a part of our
Christmas dinners when I was growing up. I remember this being my absolute
favourite part of the meal. Yep, this recipe is so good it convinced a kid to
love vegetables.

\Ing{1 large head cauliflower}
Steam cauliflower head until tender crisp.

\Ing{2 cups shredded cheddar cheese}
\Ing{\fr{1}{4} cup mayonnaise}
\Ing{2 tbsp.~Dijon mustard}
\Ing{2 tbsp.~Parmesan cheese}
Combine ingredients in a mixing bowl until relatively homogeneous.

\newstep Spoon mixture over cauliflower head, spreading evenly.

\newstep Bake in a casserole dish at 425\0F for 10 minutes.
\end{recipe}

\section{Dessert}
\label{sec:findlay-dessert}

\clearpage
